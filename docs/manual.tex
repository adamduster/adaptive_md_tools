\documentclass{article}
\title{Adaptive MD Tools}
\author{Adam Duster}
\date{\today}
\newcommand{\mb}[1]{\mathbf{#1}}
\newcommand{\mr}[1]{\mathrm{#1}}
\usepackage{amsmath}
\usepackage[margin=1in]{geometry}
\usepackage{booktabs}
\usepackage{hyperref}
\setlength{\heavyrulewidth}{1.5pt}
\setlength{\abovetopsep}{4pt}
\usepackage{enumitem}
\usepackage{microtype}
\usepackage[nottoc]{tocbibind}
% Bibliography
\usepackage[sort&compress,numbers,super]{natbib}
% ACS style
\bibliographystyle{achemso}
\begin{document}
\maketitle
\begin{abstract}
Adaptive MD Tools is a package that aims to help with getting partitions from completed MD simulations.
The user supplies a topology file and trajectory file, and then selects the active zone center.
The active zone center can be an individual atom, or one of several proton tracking algorithms including our indicator and the modified center of excess charge (mCEC).
The program can then create a new trajectory where the active center is placed in the center of the box and the other atoms are wrapped around it based on supplied unit cell dimensions for an NVT simulation.
The user can calculate the partitions from the permuted AP algorithm for a given permute order and buffer zone radius, and output them as \texttt{.xyz} files to a folder.
\end{abstract}

\section{Installation}
This program is a python package with dependencies on MDAnalysis, Numba, and NumPy. I recommend that the user create a Conda environment specifically for using this package. Then they can install the dependencies and use pip to install this program.

\paragraph{Prerequisites}

Python3 via Conda

\paragraph{Directions}
\begin{enumerate}
\item Clone the package from GitHub into the specified directory

\texttt{git clone \url{https://github.com/adamduster/adaptive_md_tools}}

\item Create a new Conda environment with the name "qmmm\_ap\_tools"

\texttt{conda create --name qmmm\_ap\_tools python=3}

\item Enter the Conda environment

\texttt{ conda activate qmmm\_ap\_tools }

\item Change to the root directory of the downloaded package

\texttt{cd adaptive\_md\_tools}

\item Install the package using pip

\texttt{pip install -e .}

\item The installation should now be complete!
You can call the file \texttt{AdaptiveMD.py} to run the program!
\end{enumerate}

\section{Usage}
The program can be invoked by typing:

\texttt{AdaptiveMD.py -i [your input file] > [your output path]}

As stated in the abstract, this program has a couple functions with keywords that need to be specified together in the input file.
Examples for various applications are given in the \texttt{examples} folder found within the module directory and these tests are described in section \ref{sec:tests}.
However, a brief walk through of the input file is given here.
For the full list of keywords and more detail, see section \ref{sec:keywords}.

\paragraph{All Input Files}
All input files must specify the coordinates, topology, and output coordinates with the following keywords:

\begin{enumerate}
	\item \texttt{coordinates}
	\item \texttt{topology}
	\item \texttt{output\_coords}
\end{enumerate}


\paragraph{Simulation Settings}
To wrap simulations, the dimensions of the period box must be specified using the \texttt{dimensions} keyword. The keyword \texttt{nowrap} can be used to prevent wrapping.

\begin{enumerate}
	\item \texttt{dimensions}
	\item \texttt{nowrap}
	\item \texttt{stride}
\end{enumerate}

\paragraph{AP Settings}
To accomplish this, the following variables can be specified:
\begin{enumerate}
	\item \texttt{groups\_file}
	\item \texttt{active\_radius}
	\item \texttt{buffer\_radius}
	\item \texttt{pap\_order}
	\item \texttt{donor\_index} to set active center as atom	
\end{enumerate}

\paragraph{Writing Partitions}
The following variables are relevant:
\begin{enumerate}
	\item \texttt{write\_partitions} (to flag for writing partitions)
	\item \texttt{write\_n\_steps}
	\item \texttt{elements\_file} (if \texttt{.xyz} files are to be written)
	\item \texttt{write\_prefix}
	\item \texttt{ap\_write\_probability} to randomly write partitions as \texttt{xyz} files	
\end{enumerate} 

\paragraph{Indicator Variables}
The following variables are relevant:
\begin{enumerate}
	\item \texttt{donor\_index} to set donor for calculation
	\item \texttt{rdh0}
	\item \texttt{proton\_types}
	\item \texttt{ind\_method}
	\item \texttt{no\_pre\_topo\_change}
	\item \texttt{varpower\_s} for Variable Power method
	\item \texttt{varpower\_t} for Variable Power method
	\item \texttt{softmax\_a} for Softmax method
	\item \texttt{softmax\_zmax} for Softmax method
	\item \texttt{outind} to write indicator location(s) to an xyz file
\end{enumerate}

\paragraph{mCEC Variables}
The following variables are required:
\begin{enumerate}
	\item \texttt{mcec}
	\item \texttt{mcec\_g} (for groups with multiple protonatable residues)
\end{enumerate}

\section{Methodology}
Our program contains indicators based on an original implementation in ref\cite{Wu2013} for proton transfer through water, but extended to monovalent anions in ref\cite{Garza2019}, and groups with multiple protonatable sites in ref.\cite{Duster2019}
There are also implementations of the mCEC\cite{Konig2006} and EPI\cite{Watanabe2020} methods.
The permuted adaptive partitioning (PAP) details can be found in ref.\cite{Pezeshki2015}

\subsection{Indicators}
This package contains a variety of proton indicators developed by our group,\cite{Wu2013,Pezeshki2015} as well as the center of excess charge (mCEC) developed by K\"onig et al.\cite{Konig2006}
These indicators are developed to describe proton transfer in a variety of conditions.
The location of these indicators can be appended to any arbitrary trajectory and used as the center of the active zone for the AP functionality of the package.
Originally, these equations were only meant to describe proton transfer through water.\cite{Wu2013,Pezeshki2015}
Later this was extended to any molecule with one protonatable site (e.g. monoatomic anions).\cite{Garza2019}
These sets of equations are described in section \ref{ss:original_indicator}.

It became necessary to describe proton transfer through molecules or residues with multiple protonation sites to study more complicated systems with adaptive partitioning, 
We call these groups donor groups if they are in possession of the excess proton, or acceptor groups if the atoms could be protonated. 
There are many different attempts at solving this problem implemented within this package.
Arguably only two are useful: the original indicator (Indicator0 here) and the indicator mentioned in our upcoming publication as "Indicator B" (Indicator4 here).
In fact, as the Indicator4 reduces to Indicator0, there may even be no reason to use Indicator4.
Indicator1 is well tested and may also be used, but based on the results of the publication this is not recommended.

\subsubsection{Original Indicator}\label{ss:original_indicator}
This is implemented as Indicator0.
The position of the proton  indicator is a linear combination of the coordinates of the donor $D$ and all possible acceptors $A_j$ within a predefined enlisting radius $r_\mathrm{\textsc{list}}$ from $D$:
\begin{equation}
\mathbf{X}_I = \frac{1}{g_I} \left( \mathbf{X}_D + \sum^{J}_{j=1} \sum^{M}_{m=1} g \left( \rho_{mj} \right) \mathbf{X}_{A_j} \right)
\end{equation}
Here, $\mathbf{X}_D$ and $\mathbf{X}_{A_{j}}$ are the Cartesian coordinates of the donor oxygen and the $j$-th possible acceptor oxygen respectively, $g ( \rho_{mj} )$ is the weight function associated with $\rho_{mj}$ (the ratio of the projected donor-acceptor vector), $M$ is the total number of $H_m$ (hydrogen atoms bonded to $D$), $J$ is the total number of possible acceptors within a radius of size $r_\mathrm{\textsc{list}}$ and $g_I$ is a normalization factor.
The ratio $\rho_{mj}$ is a metric for how close a given atom $H_m$ is to the donor $D$ versus a possible acceptor $A_j$ and is calculated according to the following equation:
\begin{equation}
\rho_{mj}=\frac{ \mathbf{r}_{\mathrm{DH}_m} \cdot \mathbf{r}_{\mathrm{DA}_j} } { | \mathbf{r}_{\mathrm{DA}_j} |^2 }
\end{equation}
where $\mathbf{r}_{\mathrm{DH}_m} = \mb{X}_{\mathrm{H}_m} - \mb{X}_\mr{D}$ and $\mb{r}_{DA_j} = \mb{X_{A_j}} - \mb{X}_\mr{D}$.
The linear combination of coordinates is normalized by $g_I$, which is calculated according to equation \ref{eq:gI} using the weight function described in equation \ref{eq:gofx_weight}:
\begin{equation}\label{eq:gI}
g_I = 1 + \sum^J_{j=1} \sum^M_{m=1} g ( \rho_{mj} )
\end{equation}

\begin{equation} \label{eq:gofx_weight}
g(x) = \begin{cases}
0
& \mathrm{if}\ 1 \leq x \\
-6x^5 + 15x^4 - 10x^3 + 1
& \mathrm{if}\  0 \leq x < 1 \\
1
& \mathrm{if}\ x < 0 
\end{cases}
\end{equation}

The weight function $g ( \rho_{mj} )$ depends on the reduced variable $x$, which is now determined by:
\begin{equation}\label{eq:xmj}
x = x ( \rho_{mj} ) = 1 - \frac{\rho_{mj} - \rho^0_{mj}}{\rho_{\mathrm{max}} - \rho^0_{mj}}
\end{equation}
\begin{equation}
\rho_{\mathrm{max}} = \frac{r^0_{\mathrm{DH}}}{r^0_{\mathrm{DH}}+r^0_{AH}}
\end{equation}
where $\rho^0_{mj} = \left( \frac{r^0_{DH}}{r_{\mathrm{\textsc{list}}}} \right)$, $r^0_{DH}$ is a parameter that is set to slightly larger than the equilibrium $D-H$ bond distance (e.g. 1.00 \r{A} in a hydronium ion) to reduce the sensitivity to the $D-H$ vibrations near equilibrium, $r^0_{AH}$ is similarly a parameter for the $A-H$ bond distance,
$r_{\mathrm{\textsc{list}}}$ is the threshold distance for enlisting possible acceptors, and $\rho_{\mathrm{max}}$ represents the percentage of the distance that a transferring proton needs to travel from the donor towards an acceptor before the donor and acceptor switch their status.
In this work, the parameters $r^0_{\mr{DH}}$ or $r^0_{\mr{AH}}$ are based on gas-phase geometries optimized at the B3LYP/6-31G* level for the protonated species and rounded up to the nearest 0.1 \AA.

\paragraph{Switching}
To reiterate, Donor switching occurs theoretically at $\rho = \rho_{\mr{max}}$.
Theoretically, switching is time-reversible in this case.
It should be noted however, that switching is non time-reversible in most practical situations. 
However, in the case of finite time steps, switching will occur when $\rho \geq \rho_{\mr{max}}$.
In cases of proton transfer between two species, this is typically not an issue as the location of the indicator will be very similar in both cases.
This is because the normalized weight of the donor will be equal to one, and the weight of the acceptor is approximately equal to 1 around that time.
Unfortunately there is a nightmare scenario which occurs in situations with concerted proton transfer.
Here a proton can travel part of the way down a water wire belong to the original donor.

\subsection{Heuristic for Molecular Topology Update for Intramolecular Reactions}\label{ss:topo_change}
Updating the molecular topology on-the-fly is important for adaptive QM/MM simulations as the H+ is transported along the path, because if molecules in the active zone later travel into the environmental region, the simulation will contain serious artifacts.
While the original indicator could deal with intermolecular topology changes, it was necessary to add additional rules for dealing with intramolecular topology changes.
One example is the tautomerization of GLU-, which sees H+ transfer between the two atoms.
The heuristic is as follows.
For each donor $k \in K$, and for each other member of the donor group $j \in K, j \ne k$ and for each proton $m$ bonded to $k$, we calculate the following ratio:
\begin{equation}\label{eq:ratio}
F = \frac{r_{km}}{r_{km} + r_{jm}}
\end{equation}
Here $r_{km}$ and $r_{jm}$ are the distances between the proton and the $k$-th or $j$-th atom respectively.
If $F > 0.5$, then the $m$-th proton is closer to the $j$-th member of the donor group than the atom it is currently bonded to.
The topology is then revised by deleting the bond between $k$ and $m$, and adding another bond between $j$ and $m$.
The angles and dihedrals will also be adjusted accordingly.

This can cause discontinuities if used with the original indicator and any indicator based on Indicator4, because in those methods, the specific bonds between a donor in a donor group and its hydrogens are used to calculate the projection vectors.
In Indicator1 however, this is no issue as all protons are considered bonded to the center of geometry of the donor group.

\subsection{Indicator A (\texttt{ind\_method 1})}\label{ss:ind_a}
This extension to the original indicator equation considers each donor or acceptor group as a singular unit with its location determined by its center of geometry.
For a donor group, all protons bonded to the individual protonation sites are considered to be bonded to the COG of the group instead.
The projection vectors between the donor and acceptors are calculated from the COG of the donor group to the COG of an acceptor group.
Therefore the location of the donor group becomes:
\begin{equation}\label{eq:donor_cog}
\mb{X}_{\mr{D}} = \frac{1}{K}\sum_k^K \mb{X}_{\mr{D}_k}
\end{equation}
Likewise the location of each acceptor is expressed as:
\begin{equation}
\mb{X}_{\mr{A}} = \frac{1}{K}\sum_k^K \mb{X}_{\mr{A}_k}
\end{equation}
where $k$ is the $k$-th protonatable site of the acceptor group.

This algorithm requires very few changes to the definition for the H+ indicator.
The geometric or mass center is the reference point for the searching of possible acceptors (when the side chain acts as the donor) and for the calculations of $g(\rho_{mj})$.
The parameter $r^0_{DH}$ are again determined based on B3LYP-optimized geometries of gas-phase models, but from the geometric center of the residue to the proton.
Due to the large size of the side chain functional groups, $r^0_{DH}$ is elongated in order to make sure that all relevant acceptors are included in the equation.
The parameters suggested for various amino acids are shown in column 3 of Table \ref{tab:indparams}.

\begin{center}
\begin{table}\label{tab:indparams}
\caption{Suggested Parameters for Indicator Algorithms}
\begin{tabular}{c c c c}
\toprule
                    & Original Algorithm   & Algorithm A & Algorithm B \\
\midrule
Donor Location      & $\mr{Site}^\alpha$ & $\mr{Center}^\beta$ & $\mr{Site}^\alpha$           \\
Reference point for Acceptor Search and Projections & $\mr{Site}^\alpha$ & $\mr{Center}^\beta$ & $\mr{Site}^\alpha$           \\
 & \multicolumn{3}{c}{$r^0_{DH}$} \\
Lysine       & 1.0 & 1.0 & 1.0 \\
Aspartate Glutamate & 1.0 & 1.9 & 1.0 \\
Histidine & 1.0 & 2.1 & 1.0 \\
Arginine & 1.0 & 2.1 & 1.0 \\
\bottomrule
\end{tabular}
\end{table}
\end{center}




\subsection{Indicator B (ind\_method 4)}\label{ss:ind_b}
Indicator B in 2019 publication by Duster and Lin,\cite{Duster2019} considers the side chain as a whole, but it recognizes that the multiple protonatable sites may be situated in different local environments and thus participate differently.
Therefore, shile all protonatable sites of the side chain act collectively as donors, they are treated independently in the search of possible acceptors and the subsequent calculations of weights for these acceptors (Fig. 2c).
However, the center of mass of the group is used in the final linear combination for the position (equation \ref{eq:ind_b}, which allows the charge to be delocalized over a group of protonatable sites.
More specifically, a local search of possible acceptors is carried out for each protonatable site $\mr{D}_k$, and the indicator position $\mb{X}_\mr{I}$ is computed by:
\begin{equation}\label{eq:ind_b}
\mathbf{X}_\mr{I} = \frac{1}{g_I} \left( \mathbf{X}_\mr{D} + \sum^K_{k=1} \sum^J_{j=1}   \sum^M_{m-1} g(\rho_{mjk}) \mathbf{X}_{\mr{A}_{\mr{COG}}}  \right) 
\end{equation}
Here, $k$ is the index for the protonatable sites of the side chain, $\mathbf{X}_{\mr{D}_k}$ is the position of the $k$-th protonation site $\mr{D}_k$, $M_k$ is the number of covalently bonded H atoms to the atom at $\mr{D}_k$, $\mathbf{X}_{\mr{A}_{\mr{COG}}}$ is the center of geometry of the protonatable sites in the group that contains the $j$-th acceptor  and $\rho_{mkj}$ and $g_{mkj}$ are the generalized projection and normalized weights for $\mr{D}_k$ respectively. 
The donor center of geometry $\mathbf{X}_\mr{D}$ is calculated using equation \ref{eq:donor_cog}.
However, the location of the individual protonation site $\mr{D}_k$ is used as the reference point for enlisting possible acceptors and calculating the projection ratio.
The $r_\mathrm{\textsc{list}}$ and $r^0_{DH}$ parameters should use the same values as they would with the original algorithm (see column 4 of Table \ref{tab:indparams}).
The entire sidechain is represented by the geometric center so that when there are no acceptors, the center of geometry is the location of the proton indicator.

The normalization constant is adjusted to accommodate the extra projection vectors and is calculated by:
\begin{equation}\label{eq:ind_b-normalization}
g_i = 1 + \sum_k^K \sum_j^J \sum_m^{M_k} g_{kjm}(\rho_{kjm})
\end{equation}
Note that the geometric center is given a weight of 1.

\subsubsection{Variation -- Increase Weighting of the Likely Acceptors \texttt{(ind\_method 6)}}\label{ss:indicator6}
The main origin of discontinuities in the indicator's position occurs when multiple projection vectors are pointed towards acceptors in the opposite directions.
These projections cancel each other out and the location of the indicator ends up on the donor instead of along the D-H bond length.
This correction tries to weight the most likely projection even further using a logit to reduce the degree of this cancellation.

Here the equation \ref{eq:ind_b} is modified by replacing:

\begin{center}
$g_{kjm}(\rho_{kjm})$ with $\exp \left[ g_{kjm}(\rho_{kjm}) \right]$

and

equation \ref{eq:ind_b-normalization} with $g_i = e + \sum^K_{k=1} \sum^J_{j=1}   \sum^M_{m-1} g(\rho_{mjk})$
\end{center}

\paragraph{Brief Description of Performance}
This improves the behavior of the indicator at the limit of proton transfer but renders the indicator more sensitive to vibrations around the equilibrium geometry.
The addional mathematical complexity of the indicator may not warrant its use.

\subsubsection{Variation -- Change $\mb{X}_{\mr{D}}$ to a Weighted Coordinate \texttt{(ind\_method 9)}}\label{ss:indicator9}
This method also tried to solve the same problem as the first but with a different approach.
Here the coordinate $\mb{X}_{\mr{D}}$ in equation \ref{eq:ind_b} was changed from the geometric center of the protonatable sites to the following:
\begin{equation}
\mb{X}_{\mr{D}} = \sum_k^K w_k \mb{X}_{\mr{D}_k}
\end{equation}
\begin{equation}
w_k = \frac{\sum_{m \in M_k} r_{km} - r^0_{DH} }{\sum_k^K r_{km} - r^0_{DH}}
\end{equation}
The idea was that the location of the indicator should be weighted further towards donor groups that had bonds that were further from equilibrium.
In practice, it seems theoretically very unsatisfactory as it just makes everything more complicated and invalidates the correction in the original equation.
This should not be used.

\subsubsection{Intramolecular $\rho_{kjm}$ Included \texttt{(ind\_method 7)}}\label{ss:indicator7}
This extension aimed to solve the problem of proton tautomerization by including the other atoms in a donor group as possible acceptors.
Thus, the projection weight would be calculated between the $m$-th proton bonded to the $k$-th other protonatable site, and the normalized projection weight is multiplied by the $k$-$j$ vectors and added just like if they were a normal acceptor.

This method performed the same as Indicator B when these projection vectors were negative and significantly worse when they weren't.


\subsection{Variable Power}\label{ss:indicator12}
The variable power is an improvement to the original method in section \ref{ss:original_indicator} that reduces the displacement of the indicator for proton transfer reactions with one protonatable site.
This method modifies the fifth order spline in equation to add the exponent $\beta$ to the spline:

\begin{equation} \label{eq:gofx_vpow}
g(x) = \begin{cases}
0
& \mathrm{if}\ 1 \leq x \\
\left( -6x^5 + 15x^4 - 10x^3 + 1 \right) ^ \beta
& \mathrm{if}\  0 \leq x < 1 \\
1
& \mathrm{if}\ x < 0 
\end{cases}
\end{equation}

where

\begin{equation}
\beta = s_1 \left( x(\rho_{mj}) - x_{\mathrm{min}} \right) + s_2
\end{equation}

Here, $x_{\mathrm{min}}$ is the smallest $x(\rho_{mj})$ (equation \ref{eq:xmj}) at a given time step.
The parameters $s_1$ and $s_2$ are empirical and we recommend choosing $s_1 = 4$ and $s_2 = 6$.
For more information, please refer to our upcoming publication "Improved Indicator Algorithms for Tracking a Hydrated Proton as A Local Structural Defect in Grotthuss Diffusion in Aqueous Solutions" when published.

Note that the current implementation only works with monovalent anions and this has not been implemented past that.

\subsection{Softmax}\label{ss:indicator13}
The softmax method is unpublished, but was created with the goal of forcing the original indicator to be continuous by weighting all of the $g_{mj}$ using a softmax probability function.
At the limit of transfer, the $g_{mj}$ of the acceptor with $\rho = 0.5$ will be equal to 1, while all of the other weights will be equal to 0.

First, the variable $z$ is calculated for each of the acceptors:

\begin{equation}
z_{mj} = g(\rho_{mj}) / (1 + \epsilon - g(\rho_{mj}) )
\end{equation}

Here, $g(\rho_{mj})$ is calculated from equation \ref{eq:gofx_weight} and $\epsilon$ is machine epsilon or some small factor to prevent a divide by zero error.
Then, these variables are converted to a softmax probability by calculating:

\begin{equation}
\sigma_{mj} = \frac{\exp(a z_{mj}) - 1}{\sum_m \sum_j \left[ \exp(a z_{mj}) - 1 \right] }
\end{equation}

where the parameter $a$ is used to scale the exponent.
We recommend using $a = 0.05$
Then the final position of the indicator is calculated as

\begin{equation}\label{eq:softmax-final}
\mathbf{X}_I = \frac{1}{g_I} \left( \mathbf{X}_D + \sum^{J}_{j=1} \sum^{M}_{m=1} g \left( \rho_{mj} \right) \sigma_{mj} \mathbf{X}_{A_j} \right)
\end{equation}

Again, here $g \left( \rho_{mj} \right)$ ensures the contribution of the acceptor smoothly changes from zero to one as the acceptor enters $r_{\mathrm{list}}$ , and $\sigma_{mj}$ ensures that the probability of the acceptor at the limit of transfer is equal to one, and the probability of all other acceptors are effectively zero.
In the implementation within the code, we introduce another variable $z_{\mathrm{max}}$ to avoid numerical errors due to the large exponent.
If $z > z_{\mathrm{max}}$ then we simply set $z = z_{\mathrm{max}}$.
This approximately corresponds to $g(\rho_{mj}) = 0.99$.

While this is a mathematically smooth function, there can be large changes in the indicators between frames if different acceptors rapidly become the most likely acceptor.
Thus, it can perform empirically worse than the variable power scheme mentioned above.

Note that the current implementation only works with monovalent anions and this has not been implemented past that.

\subsection{mCEC}
Proposed by K\"onig et al.,\cite{Konig2006}, the location of the mCEC $\zeta$ can be expressed as the sum of the positions of the protons $\mb{r}^{H_i}$ and the coordinates of each acceptor $\mb{r}^{X_j}$ multiplied by the amount of protons the acceptor is coordinated to in in its reference state.
\begin{equation}\label{eq:mcec}
\mb{\zeta} = \sum_{i=1}^{N_H} \mb{r}^{H_i} - \sum_{j=1}^{N_X} w^{X_j} \mb{r}^{X_j} + \sum_{i=1}^{N_H} \sum_{j=1}^{N_X} w^{X_j} f_{sw}(d_{H_i,X_j}) + \zeta`
\end{equation}
The term $f_{sw}$ is a switching function probosed by Chakrabarti al.\cite{Chakrabarti2004} that is applied to the distance between the $i$-th hydrogen and the $j$-th acceptor.
\begin{equation}\label{eq:chakra}
f_{sw}(d) = \frac{1}{1 + \exp[(d-r_{sw})/d_{sw}]}
\end{equation}
where $r_{sw}$ and $d_{sw}$ are adjustable parameters that correspond to the distance where the smoothing function is equal to 0.5 and the slope of the smoothing function respectively.
Here, $\zeta`$ is a correction term that accounts for groups of atoms with multiple protonatable sites. It can be expressed as:
\begin{equation}
\mb{\zeta}` = \sum_{g \in G} \sum_{X_i\in g}\sum_{X_j \in g} \frac{N_p}{N_g} (\mb{r}^{H_i} - \mb{r}^{X_j})
\end{equation}
where $g$ is a residue with multiple protonation sites in the set of all such residues $G$, $X_i$ and $X_j$ are protonatable sites within the $g$-th residue, $N_p$ is the number of protons bound to the group at the reference state and where $N_g$ is the total number of these sites in the $g$-th residue. 
The multiplier $m$ is a differentiable maximum function times a constant related to the number of protons in the residue:
\begin{equation}
m = \frac{\sum_{i=1}^{N_H} f_{sw}(d_{H_i,X_j})^{16}}{\sum_{i=1}^{N_H}  f_{sw}(d_{H_i,X_j})^{15}}
\end{equation}
For acceptor atoms that are also members of groups with multiple protonation sites, it is required that:
\begin{equation}
w^{X_j} = \frac{N_p}{N_g}
\end{equation}

\paragraph{Usage Notes}
In the implementation within this code, the mCEC cannot be used alone and must be used with the one of the other indicator based methods.
Those track the topology.
The mCEC is added as the last atom in the output coordinates (indicator 4 will be output as well).
Test 2 contains an example of how to use the mCEC.

\subsection{EPI}
The EPI can be calculated as described in ref\cite{Watanabe2020} with some minor changes.
Please see the reference for the equations. In the published method, another coordinate is added as well as the EPI coordinate.
It is then possible to search around the location of that coordinate to reduce the computational complexity of the algorithm.
Unfortunately, this is not implemented and all atoms in the system are considered when calculating the location of the EPI.

Please note that this implementation is experimental, and while it should be correct no results with it have been published.

\paragraph{Usage Notes}
To use the EPI, simply add the keyword \texttt{epi} to the input file.
This has the same requirement as the mCEC to add they keyword \texttt{ind\_method 4} so that the topology is correctly updated as the position of the EPI propagates.
The adjustable EPI parameters are set to the ones listed in the paper: $c = 0.0$, $r_o = 1.3$, and $d = 0.129$.
Note that the paper may incorrectly refer to $d$ as $\alpha$.
The smoothing function parameters are set to those defined in the appendix of ref\cite{Watanabe2020}.

\section{Input File}

\subsection{Keywords}\label{sec:keywords}
\begin{description}[style=unboxed, labelwidth=\linewidth, font =\sffamily\itshape\bfseries, listparindent =0pt, before =\sffamily]

\item[active\_radius (float)]
The active radius for adaptive partitioning calculations.

\item[allow\_hop (int to bool)]
If the indicator is calculated, indicator hops will only be allowed if this is set to 1.

\item[buffer\_radius (float)]
The buffer radius for adaptive partitioning calculations.

\item[coordinates (path to coordinate file)]
This keyword allows the user to supply the coordinates file for the program. It is required. The suffix of the file will indicate to MDAnalysis what kind of coordinate file it is.

\item[dcd\_pbc 1]
This keyword doesn't do anything. Some of the files used in previous research projects have it, so this is kept to ensure those files are compatible with this version of the code.

\item[dimensions (float) (float) (float)]
The x, y, and z dimensions of the cell for wrapping the trajectory.
The cell is assumed to have constant volume.
\item[pap\_order (int)]
The truncation order for the PAP method.
\item[donor\_index (1-based integer of donor index)]
This keyword is used to specify the donor at step 1.

\item[dsw (float)]
This sets the $d_sw$ parameter in the Chakrabati smoothing function in equation \ref{eq:chakra}.

\item[elements\_file (file type) (path to elements file)]
This keyword is used to specify the file with the element symbol for each of the atoms in the system. It is needed to write the .xyz files if the partitions are to be output.

\item[groups\_file (path to group file)]
This keyword is used to specify the AP groups file. It is needed for calculating the partitions during an AP simulation. It is updated as proton transfer happens.

\item[ind\_method (int)]
The indicator method to use for the calculation.
It can be set to:

0 --- Original indicator. See section \ref{ss:original_indicator}

1 --- Indicator A. See section \ref{ss:ind_a}.

2 --- Defunct, do not use.

3 --- Defunct, do not use.

4 --- Indicator B. See section \ref{ss:ind_b}.

5 --- Defunct, do not use.

6 --- Indicator method described in \ref{ss:indicator6}

7 --- Indicator method described in \ref{ss:indicator7}. Should not be used.

9 --- Indicator method described in \ref{ss:indicator9}

11 --- Indicator method described in \ref{ss:indicator11}

12 --- Variable power indicator as described in \ref{ss:indicator12}

13 --- Softmax indicator as described in \ref{ss:indicator13}

\item[ind\_output\_freq (integer)]
The frequency in steps to output the indicator location into the xyz file.

\item[indicator\_verbose]
Print out tons of indicator info.

\item[mcec (atom type 1) (float reference state 1) (atom type 2) (reference state 2) ..]
This keyword turns on the mCEC.
Each atom type that is to be included as an acceptor should be specified here followed by its reference state.
All atom types must also have an associated 

\item[mcec\_g  (1-based index),(1-based index),(...),(int reference state) ...]
This keyword turn on the mCEC correction for groups with multiple protonatable sites.
Each group is represented by a string with atom indices separated by commas followed finally by the reference protonation state for the group.
For example, the entry:

\texttt{mcec\_g\ \ \ 5,8,7,4\ \ \ 18,19,0\ \ \ 35,36,1}

specifies three groups with multiple protonatable sites.
The first group is contains atom indices 5,8, and 7.
It has 4 protons in its reference state.
The second group contains atom indices 18 and 19 and has no protons in its reference state.

\item[no\_pre\_topo\_change]
Disable the topology changing heuristic for intramolecular proton transfer found in equation \ref{eq:ratio}.

\item[nowrap]
Don't wrap the simulation around the center of the periodic box or active zone.

\item[proton\_types (type 1) (type 2) (type 3)]
The proton types to be considered for mCEC and indicator calculations.
Protons will only be considered bonded to the donor atoms if they have the specified type.

\item[rdh0 (atom type 1) (float) (atom type 2) (float) ...]
This keyword sets the rdh0 parameters for each atom \textbf{type}.
An example of using this code is:

\texttt{rdh0   N 1.0   OT 1.0   CLA 1.4}

\item[rlist (float)]
This sets the $r_\mathrm{\textsc{list}}$ parameter.

\item[rsw (float)]
This sets the $r_sw$ parameter in the Chakrabati smoothing function for the mCEC in equation \ref{eq:chakra}.

\item[stride (int)]
This keyword allows the user to skip every $n$ frames for the trajectory analysis.

\item[structure (path to structure file) (structure type)]
This keyword allows the user to specify the topology of the system. When doing indicator calculations with proton transfers, only .mol2 topology files can be used. Otherwise, any file type MDAnalysis accepts is fine.

\item[verbose 1]
Turn on the verbose flag. Currently it must be set to 1.

\item[write\_partitions (path to folder to write partitions) (file type)]
Write the AP partitions that would be in the trajectory to the specified folder. The files will have the extension given by the file type argument to the keyword. Note that \texttt{xyz} is the most tested format, but any format supported by MDAnalysis should work if the required info is there.

\item[write\_n\_steps (int)]
Write the partitions every $n$ steps.

\item[write\_prefix (desired prefix)]
Prefix files created by the program with some output prefix to tidy up the mess that this program creates.

\item[write\_sispa (folder)]
Write the SISPA partitions for each step in the trajectory to a given folder.

\end{description}

\section{Output Files}
The output files will have the name of the input file prefixed to the filename
\begin{description}[style=unboxed, labelwidth=\linewidth, font =\sffamily\itshape\bfseries, listparindent =0pt, before =\sffamily]

\item[-indicator.log]
This file contains a list of the largest $\rho_{mj}$ in the first column and the reaction coordinate $\delta r$ in the second column.
The coordinate $\delta r$ is calculated by
\begin{equation}\label{eq:dr}
| \mathbf{X}_D - \mathbf{X}_H | - | \mathbf{X}_A - \mathbf{X}_H |
\end{equation}
where $\mathbf{X}_D$ is the location of the donor, $mathbf{X}_A$ is the location of the most likely acceptor, and $\mathbf{X}_H$ is the location of the proton with the largest $\rho_{mj}$

\item[-donor.xyz]
This file contains the location of the donor as the first atom, and 4 other atoms with a coordinate located at the origin.
This file is used to output arbitrary coordinates for the last 4 atoms for debugging and analysis.
\end{description}


\section{Tests}\label{sec:tests}

\subsection{Test 1 -- Proton Transfer through Arginine}
In this test, a proton is transfered to one N in Arginine.
Then,a proton is transfer at the other protonatable N to a water molecule.
The test folder contains the following files:

The test can be run by:

\texttt{python AdaptiveMD.py -i ind4.in}

\paragraph{Input Files}
\begin{itemize}
\item \texttt{ind4.in} -- The XYZ trajectory file to process the indicator with
\item \texttt{mol2.mol2} -- The MOL2 file with the topology and atom types of the system
\item \texttt{groups.in} -- \textsc{QMMM} groups file to allow for adaptive partitioning calculation.
\end{itemize}

\paragraph{Output Files}
\begin{itemize}
\item \texttt{ind4.dcd} -- Output trajectory that includes the indicator as the last atom.
\item \texttt{vis.mol2} -- MOL2 vile to allow for visualizing \texttt{ind4.dcd}. This was created manually, not by the program.
\item \texttt{ind4-donor.xyz} -- XYZ trajectory file that outputs the location of the donor throughout the simulation
\item \texttt{ind4-indicator.log} -- Output log file that shows the largest $\rho_{mj}$ and the reaction coordinate $\delta r$
\end{itemize}

\subsection{Test 2 -- mCEC and Indicator Transfer through Chain of Amino Acids}
This test demonstrates the location of the mCEC and indicator through the first part of proton transfer through a chain of amino acids.
The indicator is the second to the last atom (serial 52) and the mCEC location is the last atom (serial 53)

The test can be run by:

\texttt{python AdaptiveMD.py -i mcec.in > mcec.out}

\paragraph{Input Files}
\begin{itemize}
\item \texttt{mcec.in} -- The AdaptiveMD input file used to calculate the location of the indicator.
\item \texttt{mcec\_in.xyz} -- The XYZ trajectory file containing the coordinates to add the indicator to
\item \texttt{mcec\_in.mol2} -- The MOL2 file with the topology and atom types of the system
\item \texttt{groups.in} -- \textsc{QMMM} groups file to allow for adaptive partitioning calculation.
\end{itemize}

\paragraph{Output Files}
\begin{itemize}
\item \texttt{mcec.dcd} -- Output trajectory that includes the indicator and mCEC as the second to last and last atoms, respectively.
\item \texttt{vis.mol2} -- MOL2 vile to allow for visualizing \texttt{ind4.dcd}. This was created manually, not by the program.
\item \texttt{mcec-donor.xyz} -- XYZ trajectory file that outputs the location of the donor throughout the simulation
\item \texttt{ind4-indicator.log} -- Output log file that shows the largest $\rho_{mj}$ and the reaction coordinate $\delta r$
\end{itemize}

\subsection{Test 3 -- Output AP partitions for a Given System}
This test will take a water box with a protonated water (serials 4306-4309) and calculate the location of the indicator on top of the water box.
Then the AP partitions for each step of the trajectory are written as individual XYZ files to a folder.

The test can be run by:

\texttt{python AdaptiveMD.py -i calc\_partitions.inp > calc\_partitions.out}

\paragraph{Expected Output}
\begin{itemize}
\item \texttt{out\_box.dcd} -- The output coordinates with the indicator as the last atom in the trajectory.
Note that the system is translated such that the indicator is at the origin for each step.
\item \texttt{t300xyz/} -- Folder that contains the XYZ geometry for each of the output partitions
\end{itemize}

\subsection{Test 4 -- Output SISPA partitions for a Given System}
This test will take a water box with a protonated water (serials 4306-4309) and calculate the location of the indicator on top of the water box.
Then the SISPA partitions (by Field\cite{Field2017}) for each step of the trajectory are written as individual XYZ files to a folder.

The test can be run by:

\texttt{python AdaptiveMD.py -i calc\_partitions.inp > calc\_partitions.out}

\paragraph{Expected Output}
\begin{itemize}
\item \texttt{out\_box.dcd} -- The output coordinates with the indicator as the last atom in the trajectory.
Note that the system is translated such that the indicator is at the origin for each step.
\item \texttt{t300xyz/} -- Folder that contains the XYZ geometry for each of the output partitions
\end{itemize}

\subsection{Test 5 -- AP Partitions with No Indicator}
This test uses the same geometry as the previous two water boxes, but no indicator is used.
The indicator subroutines are bypassed, and one can print out the AP partitions for any given system.

\subsection{Test 6 -- Proton Transfer through Amino Acid Sidechains}
This test demonstrates proton transfer through a chain of amino acids and has been published in ref\cite{Duster2019}.
Both the mCEC and our indicator described in section \ref{ss:ind_b} were used.
Our indicator is output as the second to the last atom, and the mCEC is output as the last atom in the trajectory.


\paragraph{Input Files}
\begin{itemize}
\item \texttt{aminos.in} -- The AdaptiveMD input file
\item \texttt{aminos.dcd} -- The trajectory of the system. It can be visualized with aminos.psf
\item \texttt{aminos.pdb} -- PDB file containing the elements of they system
\item \texttt{aminos.mol2} -- The MOL2 file with the topology and atom types of the system
\item \texttt{aminos.groups} -- \textsc{QMMM} groups file which designates the representative atoms of each AP group (first atom of group).
It also contains the rest of the atoms in the group.
\end{itemize}

\paragraph{Output Files}
\begin{itemize}
\item \texttt{aminos\_out.dcd} -- Output trajectory that includes the indicator and mCEC as the second to last and last atoms, respectively.
\item \texttt{aminos\_vis.psf} -- PSF file to allow for visualizing \texttt{aminos\_out.dcd}. This was created manually, not by the program.
\end{itemize}

\paragraph{Files Created When Ran}
\begin{itemize}
\item \texttt{aminos-donor.xyz} -- XYZ trajectory file that outputs the location of the donor throughout the simulation
\item \texttt{aminos-indicator.log} -- Output log file that shows the largest $\rho_{mj}$ and the reaction coordinate $\delta r$ (equation \ref{eq:dr}).
\end{itemize}

\subsection{Test 7 -- Variable Power Scheme}
This test uses the variable power scheme to calculate the location of the indicator.

\subsection{Test 8 -- Variable Power Scheme}
This test uses the variable power scheme to calculate the location of the indicator.

\subsection{Test 9 -- Softmax Scheme}
This test uses the softmax scheme to calculate the location of the indicator.

\section{Developers Notes}
Open the website in the folder \texttt{./api\_doc} for the developer notes.
To regenerate the documentation, run the script \texttt{./make\_pdoc.sh} .

\bibliography{manual}
\end{document}

